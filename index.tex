% Options for packages loaded elsewhere
\PassOptionsToPackage{unicode}{hyperref}
\PassOptionsToPackage{hyphens}{url}
\PassOptionsToPackage{dvipsnames,svgnames,x11names}{xcolor}
%
\documentclass[
  letterpaper,
  DIV=11,
  numbers=noendperiod]{scrartcl}

\usepackage{amsmath,amssymb}
\usepackage{lmodern}
\usepackage{iftex}
\ifPDFTeX
  \usepackage[T1]{fontenc}
  \usepackage[utf8]{inputenc}
  \usepackage{textcomp} % provide euro and other symbols
\else % if luatex or xetex
  \usepackage{unicode-math}
  \defaultfontfeatures{Scale=MatchLowercase}
  \defaultfontfeatures[\rmfamily]{Ligatures=TeX,Scale=1}
  \setmainfont[]{Lato}
\fi
% Use upquote if available, for straight quotes in verbatim environments
\IfFileExists{upquote.sty}{\usepackage{upquote}}{}
\IfFileExists{microtype.sty}{% use microtype if available
  \usepackage[]{microtype}
  \UseMicrotypeSet[protrusion]{basicmath} % disable protrusion for tt fonts
}{}
\makeatletter
\@ifundefined{KOMAClassName}{% if non-KOMA class
  \IfFileExists{parskip.sty}{%
    \usepackage{parskip}
  }{% else
    \setlength{\parindent}{0pt}
    \setlength{\parskip}{6pt plus 2pt minus 1pt}}
}{% if KOMA class
  \KOMAoptions{parskip=half}}
\makeatother
\usepackage{xcolor}
\setlength{\emergencystretch}{3em} % prevent overfull lines
\setcounter{secnumdepth}{-\maxdimen} % remove section numbering
% Make \paragraph and \subparagraph free-standing
\ifx\paragraph\undefined\else
  \let\oldparagraph\paragraph
  \renewcommand{\paragraph}[1]{\oldparagraph{#1}\mbox{}}
\fi
\ifx\subparagraph\undefined\else
  \let\oldsubparagraph\subparagraph
  \renewcommand{\subparagraph}[1]{\oldsubparagraph{#1}\mbox{}}
\fi


\providecommand{\tightlist}{%
  \setlength{\itemsep}{0pt}\setlength{\parskip}{0pt}}\usepackage{longtable,booktabs,array}
\usepackage{calc} % for calculating minipage widths
% Correct order of tables after \paragraph or \subparagraph
\usepackage{etoolbox}
\makeatletter
\patchcmd\longtable{\par}{\if@noskipsec\mbox{}\fi\par}{}{}
\makeatother
% Allow footnotes in longtable head/foot
\IfFileExists{footnotehyper.sty}{\usepackage{footnotehyper}}{\usepackage{footnote}}
\makesavenoteenv{longtable}
\usepackage{graphicx}
\makeatletter
\def\maxwidth{\ifdim\Gin@nat@width>\linewidth\linewidth\else\Gin@nat@width\fi}
\def\maxheight{\ifdim\Gin@nat@height>\textheight\textheight\else\Gin@nat@height\fi}
\makeatother
% Scale images if necessary, so that they will not overflow the page
% margins by default, and it is still possible to overwrite the defaults
% using explicit options in \includegraphics[width, height, ...]{}
\setkeys{Gin}{width=\maxwidth,height=\maxheight,keepaspectratio}
% Set default figure placement to htbp
\makeatletter
\def\fps@figure{htbp}
\makeatother

\KOMAoption{captions}{tableheading}
\makeatletter
\makeatother
\makeatletter
\makeatother
\makeatletter
\@ifpackageloaded{caption}{}{\usepackage{caption}}
\AtBeginDocument{%
\ifdefined\contentsname
  \renewcommand*\contentsname{Table of contents}
\else
  \newcommand\contentsname{Table of contents}
\fi
\ifdefined\listfigurename
  \renewcommand*\listfigurename{List of Figures}
\else
  \newcommand\listfigurename{List of Figures}
\fi
\ifdefined\listtablename
  \renewcommand*\listtablename{List of Tables}
\else
  \newcommand\listtablename{List of Tables}
\fi
\ifdefined\figurename
  \renewcommand*\figurename{Figure}
\else
  \newcommand\figurename{Figure}
\fi
\ifdefined\tablename
  \renewcommand*\tablename{Table}
\else
  \newcommand\tablename{Table}
\fi
}
\@ifpackageloaded{float}{}{\usepackage{float}}
\floatstyle{ruled}
\@ifundefined{c@chapter}{\newfloat{codelisting}{h}{lop}}{\newfloat{codelisting}{h}{lop}[chapter]}
\floatname{codelisting}{Listing}
\newcommand*\listoflistings{\listof{codelisting}{List of Listings}}
\makeatother
\makeatletter
\@ifpackageloaded{caption}{}{\usepackage{caption}}
\@ifpackageloaded{subcaption}{}{\usepackage{subcaption}}
\makeatother
\makeatletter
\@ifpackageloaded{tcolorbox}{}{\usepackage[many]{tcolorbox}}
\makeatother
\makeatletter
\@ifundefined{shadecolor}{\definecolor{shadecolor}{rgb}{.97, .97, .97}}
\makeatother
\makeatletter
\makeatother
\ifLuaTeX
  \usepackage{selnolig}  % disable illegal ligatures
\fi
\IfFileExists{bookmark.sty}{\usepackage{bookmark}}{\usepackage{hyperref}}
\IfFileExists{xurl.sty}{\usepackage{xurl}}{} % add URL line breaks if available
\urlstyle{same} % disable monospaced font for URLs
\hypersetup{
  pdftitle={STA 360},
  colorlinks=true,
  linkcolor={blue},
  filecolor={Maroon},
  citecolor={Blue},
  urlcolor={Blue},
  pdfcreator={LaTeX via pandoc}}

\title{STA 360}
\usepackage{etoolbox}
\makeatletter
\providecommand{\subtitle}[1]{% add subtitle to \maketitle
  \apptocmd{\@title}{\par {\large #1 \par}}{}{}
}
\makeatother
\subtitle{Fall 2023}
\author{}
\date{}

\begin{document}
\maketitle
\ifdefined\Shaded\renewenvironment{Shaded}{\begin{tcolorbox}[enhanced, breakable, frame hidden, sharp corners, borderline west={3pt}{0pt}{shadecolor}, interior hidden, boxrule=0pt]}{\end{tcolorbox}}\fi

\hypertarget{course-description}{%
\subsubsection{Course description}\label{course-description}}

This course introduces Bayesian modeling and inference, motivated by
real world examples. Course topics include Bayes' theorem,
exchangeability, conjugate priors, Markov chain Monte Carlo (MCMC)
approximation, Gibbs sampling, hierarchical modeling, Bayesian
regression and generalized linear models. We give attention to
quantifying uncertainty and compare and contrast Bayesian methods to the
frequentist paradigm. By the end of this course students should feel
comfortable (1) writing Bayesian models and, when appropriate, (2)
sampling from the posterior using MCMC to make inference.

\hypertarget{logistics}{%
\subsubsection{Logistics}\label{logistics}}

\hypertarget{teaching-team-office-hours}{%
\paragraph{Teaching team \& office
hours}\label{teaching-team-office-hours}}

\begin{longtable}[]{@{}
  >{\raggedright\arraybackslash}p{(\columnwidth - 8\tabcolsep) * \real{0.1944}}
  >{\raggedright\arraybackslash}p{(\columnwidth - 8\tabcolsep) * \real{0.2222}}
  >{\raggedright\arraybackslash}p{(\columnwidth - 8\tabcolsep) * \real{0.1944}}
  >{\raggedright\arraybackslash}p{(\columnwidth - 8\tabcolsep) * \real{0.1944}}
  >{\raggedright\arraybackslash}p{(\columnwidth - 8\tabcolsep) * \real{0.1944}}@{}}
\toprule()
\begin{minipage}[b]{\linewidth}\raggedright
\end{minipage} & \begin{minipage}[b]{\linewidth}\raggedright
Contact
\end{minipage} & \begin{minipage}[b]{\linewidth}\raggedright
Office hours
\end{minipage} & \begin{minipage}[b]{\linewidth}\raggedright
Location
\end{minipage} & \begin{minipage}[b]{\linewidth}\raggedright
Section
\end{minipage} \\
\midrule()
\endhead
Dr.~Alexander Fisher & \url{aaf29@duke.edu} & We: 1:30pm-3:30pm & Old
Chem 207 & Lecture \\
TBD & \href{}{TBD} & Mo: 10:00am-12:00pm & Old Chem 203B & Lab 01 \\
TBD & \href{}{TBD} & Th: 9:00am-11:00am & Old Chem 203B & Lab 02 \\
\bottomrule()
\end{longtable}

\hypertarget{meetings}{%
\paragraph{Meetings}\label{meetings}}

\begin{longtable}[]{@{}lll@{}}
\toprule()
\endhead
Lecture & Tu/Th 10:05 - 11:20am & Old Chemistry 116 \\
Lab 01 & M 3:05pm - 4:20pm & Perkins LINK 087 (Classroom 3) \\
Lab 02 & M 4:40pm - 5:55pm & Social Sciences 124 \\
\bottomrule()
\end{longtable}

Course website:
\href{https://sta360-fa23.github.io/}{sta360-fa23.github.io}

\hypertarget{course-material}{%
\subsubsection{Course material}\label{course-material}}

\begin{itemize}
\item
  \href{https://pdhoff.github.io/book/}{A First Course in Bayesian
  Statistical Methods}. As a Duke student, an electronic version of the
  book is freely available to you on Springer link. Check the errata at
  the link above.
\item
  \href{/chapterSummaries.html}{Chapter summaries}
\item
  We will use the statistical software package R on homework asignments
  in this course. R is freely available at
  \url{http://www.r-project.org/}. RStudio, the popular IDE for R, is
  freely available at \url{https://posit.co/downloads/}.
\end{itemize}

\hypertarget{schedule-of-topics}{%
\subsubsection{Schedule of topics}\label{schedule-of-topics}}

Part I: The Bayesian modeling toolkit

\begin{enumerate}
\def\labelenumi{\arabic{enumi}.}
\tightlist
\item
  Review of probability
\item
  Conjugate statistical models
\item
  Semi-conjugate models and Gibbs sampling
\end{enumerate}

Part II: Statistical model building and analysis

\begin{enumerate}
\def\labelenumi{\arabic{enumi}.}
\tightlist
\item
  Multilevel models
\item
  Linear regression
\item
  Generalized linear models
\item
  Density estimation and classification
\end{enumerate}

\hypertarget{evaluation}{%
\subsubsection{Evaluation}\label{evaluation}}

\begin{itemize}
\tightlist
\item
  40\% Homework
\item
  15\% Quiz 1
\item
  15\% Quiz 2
\item
  30\% Final
\end{itemize}

If you miss either quiz 1 or quiz 2, your missing quiz grade will be
replaced by your final exam grade. You must take at least 1 quiz to pass
the course.

\hypertarget{policies}{%
\subsubsection{Policies}\label{policies}}

\textbf{Academic integrity}

By enrolling in this course, you commit to upholding Duke's community
standard reproduced as follows:

\begin{quote}
I will not lie, cheat, or steal in my academic endeavors;

I will conduct myself honorably in all my endeavors; and

I will act if the Standard is compromised.
\end{quote}

Any violations of academic integrity will automatically result in a 0
for the assignment and will be reported to the Office of Student Conduct
for further action. For the Exam and Quizzes, students are required to
work alone. For the Homework assignments, students may work with a study
group but each student must write up and submit their own answers.



\end{document}
